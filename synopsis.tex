\documentclass[a4paper, 12pt]{article}

\usepackage[margin=2cm,top=2.7cm,bottom=2.7cm]{geometry}
\usepackage{german}
\usepackage[utf8]{inputenc}
\usepackage{lastpage}
\usepackage[bookmarks, hidelinks]{hyperref}
\usepackage{fancyhdr}
\usepackage{setspace}
\usepackage{comment}
\usepackage{graphicx}
\usepackage{calc}
\usepackage{ifthen}
\usepackage{boxedminipage}
\usepackage{amsfonts}
\usepackage{amsmath}
\usepackage{enumitem}

\sloppy
\setlength{\parindent}{8pt}
\setlist[description]{itemindent=-30pt, leftmargin=50pt, itemsep=.5em}

\pagestyle{fancy}
 
\lhead{Virtual Reality 2}
\chead{Zusammenfassung}
\rhead{WS 2011/2012} 

\lfoot{Philip Müller, inf9293}
\cfoot[Seite \thepage\ von \pageref{LastPage}]{Seite \thepage\ von \pageref{LastPage}}
\rfoot{\today}
\renewcommand{\footrulewidth}{.5pt}

%\setcounter{tocdepth}{2}

\begin{document}

\tableofcontents
\pagebreak



\section{Vorgeplänkel}


\subsection{Augmented Reality}
Verknüpfung des Virtuellen mit dem Realen\\
Milgrams Realität-Virtualität-Kontinuum
\begin{itemize}
  \item Spiegelbasiertes Head Mounted Display
  \item Teildurchlässiges Display
  \item Head Mounted Display\\
    Enthält Kamera, Display und Spiegeloptik
  \item Kamera-basierte AR
  \item Virtual Retinal Display\\
    Laserprojektion ins Auge über kleinen Spiegel
  \item Beamer-basierte AR\\
    Projektion ``auf reale Welt''
\end{itemize}


\subsection{Sinne und Wahrnehmung}

\subsubsection*{Haptik}
\begin{itemize}
  \item Exterozeption (``von Außen'', ``durch spezielle Sensoren'')
    \begin{itemize}
      \item Oberflächensensibilität
      \item Schmerzwahrnehmung
      \item Temperaturwahrnehmung
    \end{itemize}
  \item Propriozeptiv (``aus eigenem Körper'')
    \begin{description}
      \item[Lagesinn] Stellung der Gelenke
      \item[Kraftsinn] Anspannungszustand Muskeln und Sehnen
      \item[Kinästhesie] Bewegungen der Gelenke
    \end{description}
\end{itemize}

\subsubsection*{Vestibulär (Innenohr)}
Mit visuellem System ``hart'' gekoppelt, wird aber bei VR selten angesprochen\\
\(\rightarrow\) Unwohlsein wg.\ inkonsistenten Reizen



\section{3D in der Computergrafik}


\subsection{``Mono''}
Stichworte:
\begin{itemize}
  \item Überlappung
  \item Größenrelation
  \item Bewegungsparallaxe
  \item Farbgradienten (Beleuchtung)
\end{itemize}


\subsection{Stereo}

\subsubsection*{3D-Wahrnehmung}
Drei Aspekte:
\begin{description}
  \item[Konvergenz] Ausrichtung der Augen je nach Entfernung des Objekts
  \item[Parallaxe] ?
  \item[Disparität] Objekte werden je nach Entfernung seitlich verschoben zu den Objekten im Fokus abgebildet.
    Dabei wird in beiden Augen in entgegengesetzte Richtungen verschoben.
\end{description}

\subsubsection*{Simulation der Disparität}
Möglich, indem über zwei verschiedene Projektionen zwei Bilder berechnet werden und jedem Auge ``sein'' spezielles Bild angezeigt wird (durch zwei Projektionsflächen oder separierbare Bilder auf einer Fläche).

\begin{description}
  \item[separate Projektionsflächen] müssen i.d.R.\ mit Kopfbewegungen mitgeführt werden
    \begin{itemize}
      \item mit ``Abschalten der Konvergenz'' (Schielen) oder
      \item so nah an den Augen, dass sich jede Fläche nur in dem Sichtfeld \emph{eines} Auges befindet (HMD).
    \end{itemize}
  \item[separierbare Bilder] auf einer Projektionsfläche
    \begin{itemize}
      \item passiv
        \begin{itemize}
          \item Farbfilter: Separation über Wellenlänge
          \item Polarisationsfilter\\
            Nachteile:
            \begin{itemize}
              \item Leinwand muss gain-Faktor größer 1 besitzen
              \item Lichtverlust
            \end{itemize}
            Polarisationsarten:
            \begin{itemize}
              \item horizontal/vertikal: Kopfdrehung verschlechtert Filterwirkung
              \item zirkular: entgegengesetzt drehend
            \end{itemize}
        \end{itemize}
      \item aktiv: Shutter-Techniken
        \begin{itemize}
          \item schnelle Projektoren notwendig oder
          \item zwei Projektoren, deren Bilder durch Shutter schnell ein- und ausgeschaltet werden können
        \end{itemize}
    \end{itemize}
  \item[separate Strahlengänge] für beide Augen
    \begin{itemize}
      \item autostereoskopische Displays
        \begin{itemize}
          \item geringe Auflösung
          \item nur spezielle Betrachtungskorridore
        \end{itemize}
      \item Laser
      \item Hologramme
    \end{itemize}
\end{description}



\section{Projektionstechnik}


\subsection{Projektor-Typen}

\subsubsection*{CRT}
Vorteile:
\begin{itemize}
  \item hohe Auflösung
  \item guter Schwarzwert
  \item beste Farbdarstellung
  \item leicht modifizierbare Bildgeometrie
\end{itemize}
Nachteile:
\begin{itemize}
  \item Kalibrierung aufwändig und regelmäßig notwendig
  \item geringe Helligkeit
  \item teuer
\end{itemize}

\subsubsection*{LCD}
Projektionslampe durchleuchtet LCD-Panel.\\
Varianten:
\begin{itemize}
  \item mehrere farbige Lichtquellen und monochrome LCD-Panels
  \item eine Lichtquelle, mehrfarbiges Panel
\end{itemize}
Vorteile:
\begin{itemize}
  \item durch Lampe heller als CRT
  \item keine Kalibration notwendig
\end{itemize}
Nachteile:
\begin{itemize}
  \item weniger farbtreu als CRT, Farben stärker quantisiert
  \item schlechter Schwarzwert
  \item geringe Auflösung, klar getrennte Pixel
  \item relativ träge
\end{itemize}

\subsubsection*{DLP}
Pro Pixel ein Mikrospiegel, die Farben entstehen durch ein Farbrad. Spiegel lenken das Licht entweder durch die Farbscheibe in die Projektionsoptik oder auf einen Licht-Absorber (binär!). Abstufungen durch Pulsweitenmodulation.\\
Varianten:
\begin{itemize}
  \item Farbradsystem
  \item 3-Chip-System: Pro Farbe ein Mikrospiegel-Chip
\end{itemize}
Vorteile:
\begin{itemize}
  \item Super hell
  \item gut Bildschärfe
\end{itemize}
Nachteile:
\begin{itemize}
  \item Breite Stege zwischen Pixeln (Spiegeln)
  \item Streulicht reduziert Kontrast (trotzdem besser als LCD!)
\end{itemize}

\subsubsection*{ILA}
Ansteuerung eines LCD-Panels ``von hinten'' statt über Leiterbahnen zwischen den Pixeln. Keine klassiche Durchleuchtung des LCD-Panels, sondern Reflektion des Lichts am Panel.\\
Varianten:
\begin{itemize}
  \item ILA: Ansteuerung durch Photoleiter-Platte, die von CRT angestrahlt wird
  \item D-ILA: Ansteuerungspanel digital statt Photoleiter und CRT
\end{itemize}
Vorteile:
\begin{itemize}
  \item keine Stege zwischen Pixeln
  \item heller als CRT
  \item weichere Pixelübergänge als beim klassischen LCD
\end{itemize}
Nachteile:
\begin{itemize}
  \item sehr teuer
  \item sehr groß
  \item Lampen mit geringer Lebensdauer
\end{itemize}

\subsubsection*{Laser}
Zeilen-/Spaltenweises Scannen der Projektionsfläche mit drei farbigen Laserstrahlen.\\
Vorteile:
\begin{itemize}
  \item Bild muss nicht scharfgestellt werden bzw. ist auf beliebige Abstände scharf (gut für nichtplanare Projektionen!)
  \item extrem hell
  \item extrem guter Kontrast
\end{itemize}
Nachteile:
\begin{itemize}
  \item riesige Gerätschaften
  \item Kosten im hohen sechsstelligen Bereich
  \item Laser-Erzeuger wartungsintesiv, kurze Lebensdauer
  \item hoher Stromverbrauch
  \item kaum Hersteller, schlechte Verfügbarkeit
\end{itemize}


\subsection{Projektionsflächen}

\subsubsection*{Rückprojektion}
\begin{itemize}
  \item Plexiglasscheiben: sehr teuer
  \item Folien: empfindlich
  \item gain \(> 1\)
\end{itemize}

\subsubsection*{Frontprojektion}
--- keine großartigen Infos hier ---

\subsubsection*{Powerwalls}
Vorteil: Auflösung und Helligkeit ``beliebig'' vergrößerbar. Viele Rechner \(\rightarrow\) super für parallelisiertes Raytracing.\\
Probleme:
\begin{itemize}
  \item Synchronisation
  \item Übergänge: Geometrie/Überlappung, Farben, Helligkeit
\end{itemize}

\subsubsection*{HMDs}
--- keine großartigen Infos hier ---

\subsubsection*{CAVE-artige Projektion}
Projektion auf große Leinwände nah am Anwender. Die Projektion kann dadurch Objekte in Lebensgröße darstellen. ``VR'' für größeres Publikum.

\subsubsection*{Responsive Workbench}
--- keine großartigen Infos hier ---

\subsubsection*{CAVEs}
\begin{itemize}
  \item 2-seitig
    \begin{itemize}
      \item stärkere Immersion ggü.\ planarer Projektion --- Anwender fühlt sich ``mehr umgeben''
      \item welche Seiten nimmt man am besten?
    \end{itemize}
  \item 3-seitig
    \begin{itemize}
      \item Field Of View fast komplett abgedeckt
    \end{itemize}
  \item 5-seitig
    \begin{itemize}
      \item Field Of View komplett abgedeckt
      \item Gefahr: Anwender fühlt sich ``eingeschlossen'', da kaum Bezug zur Realwelt
    \end{itemize}
  \item Varianten
    \begin{itemize}
      \item Zusammenlegen mehrerer Wände zu einer gekrümmten Projektionsfläche
      \item zylindrisch/sphärisch
      \item aufwändige Projektion
      \item aber: keine Knicke!
    \end{itemize}
\end{itemize}


\subsection{Mehrprojektorsysteme}
\begin{itemize}
  \item aufwändige Hardware, u.a.\ wg.\ Hardware-Sync
  \item Geometrie, Intensitäten und Farben der einzelnen Projektionen müssen angeglichen werden
  \item z.B. Shift-Optik für Geometrie-Anpassungen
  \item sonst kann durch Software reguliert werden --- auch automatisch (mit Kameras)
  \item hoher Gain-Faktor problematisch
\end{itemize}


\subsection{VRD}
Projektion direkt auf die Netzhaut durch Laser, der per ``Scanner'' (Spiegel) eine Bildmatrix auf der Netzhaut erzeugt.
\begin{itemize}
  \item leichte, alltagstaugliche Geräte
  \item Stereo-Betrieb möglich
  \item Nachteil: Blockiert der Scanner, lasert er das Auge weg
\end{itemize}



\section{Nichttriviale Projektionen}


\subsection{Grundlegende Problematik}
\begin{itemize}
  \item Rendering Pipeline erlaubt nur eine, planare mathematische Projektion
  \item Allgemeine Projektor-Projektionen benötigen mehrere mathematische Projektionen
  \item Mehrere ``konkatenierte'' math.\ Projektionen in Rendering-Pipeline nicht möglich, da der Projektionsschritt in der Pipeline auch Clipping und Filling enthält, was nur einmal ausgeführt werden soll
  \item Raytracing: Lösung, aber langsam
  \item Auslagerung von Teilaufgaben an die Rendering-Pipeline
\end{itemize}


\subsection[Ebene Projektionsflächen]{Rendering auf eine ebene Projektionsfläche}
\label{subsec:even_proj}

\subsubsection*{Raytracing}
Verfolgung von Strahlen vom realen Augpunkt durch die Pixel der Darstellungsfläche auf die virtuellen Objekte.

\subsubsection*{Rendering-Pipeline}
\begin{itemize}
  \item Definition einer virtuellen Projektionsfläche, die mit der realen identisch ist
  \item Berechnung der Projektion (3D \(\rightarrow\) 2D) in der Grafikhardware durch Matrixmultiplikation
\end{itemize}


\subsection[Allgemeine Projektionsflächen]{Rendering auf eine allgemeine Projektionsfläche}

\subsubsection*{Reines Raytracing}
Siehe Abschnitt~\ref{subsec:even_proj}

\subsubsection*{Raytracing nach Pipeline-Projektion}
\begin{enumerate}
  \item Projektion auf planare Hilfsprojektionsfläche durch Rendering-Pipeline
  \item Raytracing der Hilfsprojektionsfläche auf die reale Projektionsfläche\\
    Relativ geringer Raytracing-Aufwand!
\end{enumerate}


\subsection[Projektor und allgemeine Projektionsflächen]{Rendering auf eine allgemeine Projektor-bestrahlte Projektionsfläche}

\subsubsection*{Problematik}
Steht der Projektor nicht im Augpunkt (übliche Situation), so muss eine zusätzliche Projektion aus seiner Sicht auf die Projektionsfläche berechnet werden.

\subsubsection*{Doppeltes Raytracing}
\begin{enumerate}
  \item Raytracing der Pixel des Beamers auf die Projektionsfläche
  \item Raytracing vom Augpunkt durch projizierte Pixel auf die Szene
\end{enumerate}

\subsubsection*{Doppeltes Raytracing nach Pipeline-Projektion}
\begin{enumerate}
  \item Projektion auf planare Hilfsfläche zwischen Augpunkt und Projektionsfläche durch die Rendering-Pipeline
  \item Raytracing vom Beamer über die Projektionsfläche auf die Hilfsfläche
\end{enumerate}
Die Hilfsprojektionsfläche kann auch als Textur auf die Projektionsfläche gemapt werden, um einen Strahl pro Pixel zu sparen.

\subsubsection*{Pipeline-Projektion nach doppeltem Raytracing}
\begin{enumerate}
  \item Raytracing der Szene auf die Projektionsfläche
  \item Pipeline-Projektion der Projektionsfläche auf die Darstellungs-Hilfsfläche des Beamers
\end{enumerate}
Wie kann aus der per Raytracing projizierten Szene ein OpenGL-Objekt für die Rendering-Pipeline erzeugt werden?

\subsubsection*{Doppelte Pipeline-Projektion}
\begin{enumerate}
  \item Pipeline-Projektion der Szene auf Hilfsfläche beim Beobachter
  \item Mappen der Hilfsfläche als Textur auf die Projektionsfläche
  \item Pipeline-Projektion der Projektionsfläche auf Hilfsfläche beim Beamer
\end{enumerate}

\subsubsection*{Doppelte Rendering-Pipeline mit zwischengeschaltetem Raytracing}
\begin{description}
  \item[Variante 1] (Idee 9)
    \begin{enumerate}
      \item Pipeline-Projektion der Szene auf Hilfsfläche beim Beobachter
      \item ?
      \item ?
    \end{enumerate}
  \item[Variante 2] (Idee 9a)\\
    Planare Projektionsfläche einfacher? Nein.
  \item[Variante 3] (Idee 9b)\\
    Wenn der Projektor senkrecht auf eine planare Projektionsfläche strahlt, kann das Verfahren etwas vereinfacht werden.
    \begin{enumerate}
      \item Projektor strahlt Projektionsfläche ``direkt'' an (hier muss keine Projektion berechnet werden)
      \item Pipeline-Rendering der Szene auf die Projektionsfläche in einem Schritt
    \end{enumerate}
    Birgt Ungenauigkeiten, trifft aber auf viele VR-Systeme zu.
\end{description}


\subsection{Beamerstandortkorrektur per linearer Transformation}
Kalibrierung per Software statt physischer Ausrichtung des Projektors
\begin{itemize}
  \item Projektion von drei Kalibrierpunkten auf die Projektionsfläche
  \item Interaktives Verschieben der Punkte in Ecken der Projektionsfläche
  \item Ergebnis: drei 2D-Punkte, die mit der Projektionsmatrix verrechnet weren können
\end{itemize}
Das ist nur ein Notbehelf!


\subsection{Spiegel}
--- schließe ich aus ---



\section{Klangwahrnehmung}



\section{3D-Sound-Rendering}



\section{Haptische Wahrnehmung}



\section{Haptik-Simulation}



\section{Echtzeitschatten}


\end{document}
