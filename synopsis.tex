\documentclass[a4paper, 12pt]{article}

\usepackage[margin=2cm,top=2.7cm,bottom=2.7cm]{geometry}
\usepackage{german}
\usepackage[utf8]{inputenc}
\usepackage{lastpage}
\usepackage[bookmarks, hidelinks]{hyperref}
\usepackage{fancyhdr}
\usepackage{setspace}
\usepackage{comment}
\usepackage{graphicx}
\usepackage{calc}
\usepackage{ifthen}
\usepackage{boxedminipage}
\usepackage{amsfonts}
\usepackage{amsmath}
\usepackage{enumitem}

\sloppy
\setlength{\parindent}{8pt}
\setlist[description]{itemindent=-30pt, leftmargin=50pt, itemsep=.5em}

\pagestyle{fancy}
 
\lhead{Virtual Reality 2}
\chead{Zusammenfassung}
\rhead{WS 2011/2012}

\lfoot{Philip Müller, inf9293}
\cfoot[Seite \thepage\ von \pageref{LastPage}]{Seite \thepage\ von \pageref{LastPage}}
\rfoot{\today}
\renewcommand{\footrulewidth}{.5pt}

%\setcounter{tocdepth}{2}

\begin{document}

\tableofcontents
\pagebreak



\section{Vorgeplänkel}


\subsection{Augmented Reality}
Verknüpfung des Virtuellen mit dem Realen\\
Milgrams Realität-Virtualität-Kontinuum
\begin{itemize}
  \item Spiegelbasiertes Head Mounted Display
  \item Teildurchlässiges Display
  \item Head Mounted Display\\
    Enthält Kamera, Display und Spiegeloptik
  \item Kamera-basierte AR
  \item Virtual Retinal Display\\
    Laserprojektion ins Auge über kleinen Spiegel
  \item Beamer-basierte AR\\
    Projektion ``auf reale Welt''
\end{itemize}


\subsection{Sinne und Wahrnehmung}

\subsubsection*{Haptik}
\begin{itemize}
  \item Exterozeption (``von Außen'', ``durch spezielle Sensoren'')
    \begin{itemize}
      \item Oberflächensensibilität
      \item Schmerzwahrnehmung
      \item Temperaturwahrnehmung
    \end{itemize}
  \item Propriozeptiv (``aus eigenem Körper'')
    \begin{description}
      \item[Lagesinn] Stellung der Gelenke
      \item[Kraftsinn] Anspannungszustand Muskeln und Sehnen
      \item[Kinästhesie] Bewegungen der Gelenke
    \end{description}
\end{itemize}

\subsubsection*{Vestibulär (Innenohr)}
Mit visuellem System ``hart'' gekoppelt, wird aber bei VR selten angesprochen\\
\(\rightarrow\) Unwohlsein wg.\ inkonsistenten Reizen



\section{3D in der Computergrafik}


\subsection[Mono]{``Mono''}
Stichworte:
\begin{itemize}
  \item Überlappung
  \item Größenrelation
  \item Bewegungsparallaxe
  \item Farbgradienten (Beleuchtung)
\end{itemize}


\subsection{Stereo}

\subsubsection*{3D-Wahrnehmung}
Drei Aspekte:
\begin{description}
  \item[Konvergenz] Ausrichtung der Augen je nach Entfernung des Objekts
  \item[Parallaxe] ?
  \item[Disparität] Objekte werden je nach Entfernung seitlich verschoben zu den Objekten im Fokus abgebildet.
    Dabei wird in beiden Augen in entgegengesetzte Richtungen verschoben.
\end{description}

\subsubsection*{Simulation der Disparität}
Möglich, indem über zwei verschiedene Projektionen zwei Bilder berechnet werden und jedem Auge ``sein'' spezielles Bild angezeigt wird (durch zwei Projektionsflächen oder separierbare Bilder auf einer Fläche).

\begin{description}
  \item[separate Projektionsflächen] müssen i.d.R.\ mit Kopfbewegungen mitgeführt werden
    \begin{itemize}
      \item mit ``Abschalten der Konvergenz'' (Schielen) oder
      \item so nah an den Augen, dass sich jede Fläche nur in dem Sichtfeld \emph{eines} Auges befindet (HMD).
    \end{itemize}
  \item[separierbare Bilder] auf einer Projektionsfläche
    \begin{itemize}
      \item passiv
        \begin{itemize}
          \item Farbfilter: Separation über Wellenlänge
          \item Polarisationsfilter\\
            Nachteile:
            \begin{itemize}
              \item Leinwand muss gain-Faktor größer 1 besitzen
              \item Lichtverlust
            \end{itemize}
            Polarisationsarten:
            \begin{itemize}
              \item horizontal/vertikal: Kopfdrehung verschlechtert Filterwirkung
              \item zirkular: entgegengesetzt drehend
            \end{itemize}
        \end{itemize}
      \item aktiv: Shutter-Techniken
        \begin{itemize}
          \item schnelle Projektoren notwendig oder
          \item zwei Projektoren, deren Bilder durch Shutter schnell ein- und ausgeschaltet werden können
        \end{itemize}
    \end{itemize}
  \item[separate Strahlengänge] für beide Augen
    \begin{itemize}
      \item autostereoskopische Displays
        \begin{itemize}
          \item geringe Auflösung
          \item nur spezielle Betrachtungskorridore
        \end{itemize}
      \item Laser
      \item Hologramme
    \end{itemize}
\end{description}



\section{Projektionstechnik}


\subsection{Projektor-Typen}

\subsubsection*{CRT}
Vorteile:
\begin{itemize}
  \item hohe Auflösung
  \item guter Schwarzwert
  \item beste Farbdarstellung
  \item leicht modifizierbare Bildgeometrie
\end{itemize}
Nachteile:
\begin{itemize}
  \item Kalibrierung aufwändig und regelmäßig notwendig
  \item geringe Helligkeit
  \item teuer
\end{itemize}

\subsubsection*{LCD}
Projektionslampe durchleuchtet LCD-Panel.\\
Varianten:
\begin{itemize}
  \item mehrere farbige Lichtquellen und monochrome LCD-Panels
  \item eine Lichtquelle, mehrfarbiges Panel
\end{itemize}
Vorteile:
\begin{itemize}
  \item durch Lampe heller als CRT
  \item keine Kalibration notwendig
\end{itemize}
Nachteile:
\begin{itemize}
  \item weniger farbtreu als CRT, Farben stärker quantisiert
  \item schlechter Schwarzwert
  \item geringe Auflösung, klar getrennte Pixel
  \item relativ träge
\end{itemize}

\subsubsection*{DLP}
Pro Pixel ein Mikrospiegel, die Farben entstehen durch ein Farbrad. Spiegel lenken das Licht entweder durch die Farbscheibe in die Projektionsoptik oder auf einen Licht-Absorber (binär!). Abstufungen durch Pulsweitenmodulation.\\
Varianten:
\begin{itemize}
  \item Farbradsystem
  \item 3-Chip-System: Pro Farbe ein Mikrospiegel-Chip
\end{itemize}
Vorteile:
\begin{itemize}
  \item Super hell
  \item gut Bildschärfe
\end{itemize}
Nachteile:
\begin{itemize}
  \item Breite Stege zwischen Pixeln (Spiegeln)
  \item Streulicht reduziert Kontrast (trotzdem besser als LCD!)
\end{itemize}

\subsubsection*{ILA}
Ansteuerung eines LCD-Panels ``von hinten'' statt über Leiterbahnen zwischen den Pixeln. Keine klassiche Durchleuchtung des LCD-Panels, sondern Reflektion des Lichts am Panel.\\
Varianten:
\begin{itemize}
  \item ILA: Ansteuerung durch Photoleiter-Platte, die von CRT angestrahlt wird
  \item D-ILA: Ansteuerungspanel digital statt Photoleiter und CRT
\end{itemize}
Vorteile:
\begin{itemize}
  \item keine Stege zwischen Pixeln
  \item heller als CRT
  \item weichere Pixelübergänge als beim klassischen LCD
\end{itemize}
Nachteile:
\begin{itemize}
  \item sehr teuer
  \item sehr groß
  \item Lampen mit geringer Lebensdauer
\end{itemize}

\subsubsection*{Laser}
Zeilen-/Spaltenweises Scannen der Projektionsfläche mit drei farbigen Laserstrahlen.\\
Vorteile:
\begin{itemize}
  \item Bild muss nicht scharfgestellt werden bzw. ist auf beliebige Abstände scharf (gut für nichtplanare Projektionen!)
  \item extrem hell
  \item extrem guter Kontrast
\end{itemize}
Nachteile:
\begin{itemize}
  \item riesige Gerätschaften
  \item Kosten im hohen sechsstelligen Bereich
  \item Laser-Erzeuger wartungsintesiv, kurze Lebensdauer
  \item hoher Stromverbrauch
  \item kaum Hersteller, schlechte Verfügbarkeit
\end{itemize}


\subsection{Projektionsflächen}

\subsubsection*{Rückprojektion}
\begin{itemize}
  \item Plexiglasscheiben: sehr teuer
  \item Folien: empfindlich
  \item gain \(> 1\)
\end{itemize}

\subsubsection*{Frontprojektion}
--- keine großartigen Infos hier ---

\subsubsection*{Powerwalls}
Vorteil: Auflösung und Helligkeit ``beliebig'' vergrößerbar. Viele Rechner \(\rightarrow\) super für parallelisiertes Raytracing.\\
Probleme:
\begin{itemize}
  \item Synchronisation
  \item Übergänge: Geometrie/Überlappung, Farben, Helligkeit
\end{itemize}

\subsubsection*{HMDs}
--- keine großartigen Infos hier ---

\subsubsection*{CAVE-artige Projektion}
Projektion auf große Leinwände nah am Anwender. Die Projektion kann dadurch Objekte in Lebensgröße darstellen. ``VR'' für größeres Publikum.

\subsubsection*{Responsive Workbench}
--- keine großartigen Infos hier ---

\subsubsection*{CAVEs}
\begin{itemize}
  \item 2-seitig
    \begin{itemize}
      \item stärkere Immersion ggü.\ planarer Projektion --- Anwender fühlt sich ``mehr umgeben''
      \item welche Seiten nimmt man am besten?
    \end{itemize}
  \item 3-seitig
    \begin{itemize}
      \item Field Of View fast komplett abgedeckt
    \end{itemize}
  \item 5-seitig
    \begin{itemize}
      \item Field Of View komplett abgedeckt
      \item Gefahr: Anwender fühlt sich ``eingeschlossen'', da kaum Bezug zur Realwelt
    \end{itemize}
  \item Varianten
    \begin{itemize}
      \item Zusammenlegen mehrerer Wände zu einer gekrümmten Projektionsfläche
      \item zylindrisch/sphärisch
      \item aufwändige Projektion
      \item aber: keine Knicke!
    \end{itemize}
\end{itemize}


\subsection{Mehrprojektorsysteme}
\begin{itemize}
  \item aufwändige Hardware, u.a.\ wg.\ Hardware-Sync
  \item Geometrie, Intensitäten und Farben der einzelnen Projektionen müssen angeglichen werden
  \item z.B. Shift-Optik für Geometrie-Anpassungen
  \item sonst kann durch Software reguliert werden --- auch automatisch (mit Kameras)
  \item hoher Gain-Faktor problematisch
\end{itemize}


\subsection{VRD}
Projektion direkt auf die Netzhaut durch Laser, der per ``Scanner'' (Spiegel) eine Bildmatrix auf der Netzhaut erzeugt.
\begin{itemize}
  \item leichte, alltagstaugliche Geräte
  \item Stereo-Betrieb möglich
  \item Nachteil: Blockiert der Scanner, lasert er das Auge weg
\end{itemize}



\section{Nichttriviale Projektionen}


\subsection{Grundlegende Problematik}
\begin{itemize}
  \item Rendering Pipeline erlaubt nur eine, planare mathematische Projektion
  \item Allgemeine Projektor-Projektionen benötigen mehrere mathematische Projektionen
  \item Mehrere ``konkatenierte'' math.\ Projektionen in Rendering-Pipeline nicht möglich, da der Projektionsschritt in der Pipeline auch Clipping und Filling enthält, was nur einmal ausgeführt werden soll
  \item Raytracing: Lösung, aber langsam
  \item Auslagerung von Teilaufgaben an die Rendering-Pipeline
\end{itemize}


\subsection[Ebene Projektionsflächen]{Rendering auf eine ebene Projektionsfläche}
\label{subsec:even_proj}

\subsubsection*{Raytracing}
Verfolgung von Strahlen vom realen Augpunkt durch die Pixel der Darstellungsfläche auf die virtuellen Objekte.

\subsubsection*{Rendering-Pipeline}
\begin{itemize}
  \item Definition einer virtuellen Projektionsfläche, die mit der realen identisch ist
  \item Berechnung der Projektion (3D \(\rightarrow\) 2D) in der Grafikhardware durch Matrixmultiplikation
\end{itemize}


\subsection[Allgemeine Projektionsflächen]{Rendering auf eine allgemeine Projektionsfläche}

\subsubsection*{Reines Raytracing}
Siehe Abschnitt~\ref{subsec:even_proj}

\subsubsection*{Raytracing nach Pipeline-Projektion}
\begin{enumerate}
  \item Projektion auf planare Hilfsprojektionsfläche durch Rendering-Pipeline
  \item Raytracing der Hilfsprojektionsfläche auf die reale Projektionsfläche\\
    Relativ geringer Raytracing-Aufwand!
\end{enumerate}


\subsection[Projektor und allgemeine Projektionsflächen]{Rendering auf eine allgemeine Projektor-bestrahlte Projektionsfläche}

\subsubsection*{Problematik}
Steht der Projektor nicht im Augpunkt (übliche Situation), so muss eine zusätzliche Projektion aus seiner Sicht auf die Projektionsfläche berechnet werden.

\subsubsection*{Doppeltes Raytracing}
\begin{enumerate}
  \item Raytracing der Pixel des Beamers auf die Projektionsfläche
  \item Raytracing vom Augpunkt durch projizierte Pixel auf die Szene
\end{enumerate}

\subsubsection*{Doppeltes Raytracing nach Pipeline-Projektion}
\begin{enumerate}
  \item Projektion auf planare Hilfsfläche zwischen Augpunkt und Projektionsfläche durch die Rendering-Pipeline
  \item Raytracing vom Beamer über die Projektionsfläche auf die Hilfsfläche
\end{enumerate}
Die Hilfsprojektionsfläche kann auch als Textur auf die Projektionsfläche gemapt werden, um einen Strahl pro Pixel zu sparen.

\subsubsection*{Pipeline-Projektion nach doppeltem Raytracing}
\begin{enumerate}
  \item Raytracing der Szene auf die Projektionsfläche
  \item Pipeline-Projektion der Projektionsfläche auf die Darstellungs-Hilfsfläche des Beamers
\end{enumerate}
Wie kann aus der per Raytracing projizierten Szene ein OpenGL-Objekt für die Rendering-Pipeline erzeugt werden?

\subsubsection*{Doppelte Pipeline-Projektion}
\begin{enumerate}
  \item Pipeline-Projektion der Szene auf Hilfsfläche beim Beobachter
  \item Mappen der Hilfsfläche als Textur auf die Projektionsfläche
  \item Pipeline-Projektion der Projektionsfläche auf Hilfsfläche beim Beamer
\end{enumerate}

\subsubsection*{Doppelte Rendering-Pipeline mit zwischengeschaltetem Raytracing}
\begin{description}
  \item[Variante 1] (Idee 9)
    \begin{enumerate}
      \item Pipeline-Projektion der Szene auf Hilfsfläche beim Beobachter
      \item Raytracing durch ein Mesh auf der Hilfsfläche beim Projektor
      \item Zuweisung von Texturkoordinaten (Betrachten der Hilfsfläche beim Beobachter als Textur) auf die Vertices des Mesh
    \end{enumerate}
  \item[Variante 2] (Idee 9a)\\
    Planare Projektionsfläche einfacher? Nein.
  \item[Variante 3] (Idee 9b)\\
    Wenn der Projektor senkrecht auf eine planare Projektionsfläche strahlt, kann das Verfahren etwas vereinfacht werden.
    \begin{enumerate}
      \item Projektor strahlt Projektionsfläche ``direkt'' an (hier muss keine Projektion berechnet werden)
      \item Pipeline-Rendering der Szene auf die Projektionsfläche in einem Schritt
    \end{enumerate}
    Birgt Ungenauigkeiten, trifft aber auf viele VR-Systeme zu.
\end{description}


\subsection{Beamerstandortkorrektur per linearer Transformation}
Kalibrierung per Software statt physischer Ausrichtung des Projektors
\begin{itemize}
  \item Projektion von drei Kalibrierpunkten auf die Projektionsfläche
  \item Interaktives Verschieben der Punkte in Ecken der Projektionsfläche
  \item Ergebnis: drei 2D-Punkte, die mit der Projektionsmatrix verrechnet werden können
\end{itemize}
Das ist nur ein Notbehelf!


\subsection{Spiegel}
--- schließe ich aus ---



\section{Klang}


\subsection[Wahrnehmung]{Klang-Wahrnehmung}
\begin{itemize}
  \item Zwei Ohren
    \begin{itemize}
      \renewcommand{\labelitemii}{\(\Rightarrow\)}%
      \item Phasenverschiebung
      \item Laufzeitunterschied
      \item Lautstärkeunterschied
    \end{itemize}
  \item Reflexionen am Oberkörper/Hals verändern das Frequenzspektrum\\
    Beschrieben durch HRTF
\end{itemize}
Wahrnehmbare Parameter:
\begin{itemize}
  \item Richtung horizontal und vertikal
  \item Distanz (stark erfahrungsbasiert, relativ)
  \item Raumdimension und Beschaffenheit (durch Reflektions- und Nachhallverhalten)
\end{itemize}


\subsection[Rendering]{Sound-Rendering}

\subsubsection*{Grundlagen}
\begin{itemize}
  \item 3D-Positionierung von virtuellen Klangquellen im Raum im Gegensatz zu Stereo: ``nur links/rechts''
  \item Phasenverschiebung und Amplitudenunterschiede realisierbar
  \item HRTF rechenaufwändig aber weit verbreitet
\end{itemize}
Wellenverhalten zur Erinnerung:
\begin{description}
  \item[Beugung] eher bei niedrigen Frequenzen, kleinen Hindernissen
  \item[Reflexion] eher bei hohen Frequenzen, glatten Flächen
\end{description}
Möglichkeiten zur Klangpositionierung:
\begin{description}
  \item[Surround Sound:] Nachbildung bestimmter Bestandteile des Raumklangeffekts
    \begin{itemize}
      \item 2.0 -- 5.1 -- 7.1 -- ...
      \item Phasenverschiebung
      \item Amplitudendifferenz
    \end{itemize}
  \item[Reproduktion des Schallfeldes:] komplette Nachbildung des Raumklangs
    \begin{itemize}
      \item Vector Based Amplitude Panning (VBAP)
      \item Ambisonic
      \item Wellenfeldsynthese
    \end{itemize}
  \item[Psychoakustischer Ansatz:] Welche Signale müssten im Ohr ankommen? Wie bekommt man die da mit möglichst wenig Aufwand hin?
    \begin{itemize}
      \item binaural (Kopfhörer)
      \item transaural (Stereo-Lautsprecher)
    \end{itemize}
\end{description}

\subsubsection*{``Surround Sound''}
--- keine großartigen Infos hier ---

\subsubsection*{Vector Based Amplitude Panning}
\begin{itemize}
  \item Aufspannen einer Fläche durch min.\ 3 Lautsprecher
  \item Positionieren virtueller Klangquellen auf der Fläche durch Lautstärkeunterschiede in den einzelnen Lautsprechern
\end{itemize}
\begin{itemize}
  \renewcommand{\labelitemi}{\(-\)}%
  \item Positionierung funktioniert nur gut, wenn virtuelle Schalquelle nicht zu weit von einem Lautsprecher entfernt. Dadurch viele Lautsprecher erforderlich
  \item keine Verwendung der Phase zur Positionierung
  \item virtuelle Schallquellen können nicht in der direkten Umgebung des Hörers positioniert werden
\end{itemize}
\begin{itemize}
  \renewcommand{\labelitemi}{+}%
  \item theoretisch beliebige Anzahl und Anordnung der Lautsprecher
  \item pro virtueller Schallquelle nur drei Lautsprecher aktiv
\end{itemize}

\subsubsection*{Ambisonic}
\begin{itemize}
  \item Aufnahme- und Wiedergabetechnik
  \item Geheimnis liegt im Mixing
  \item Aufnahme der Gesamtamplitude sowie der Schwingungen den Richtungen
    \begin {itemize}
      \item oben -- unten
      \item links -- rechts
      \item vorne -- hinten
    \end{itemize}
  \item 4 Mikrofone, eins mit Kugelcharakteristik, 3 mit Acht-Charakteristik
\end{itemize}
\begin{itemize}
  \renewcommand{\labelitemi}{\(-\)}%
  \item Gute Qualität nur an bestimmter Hörposition (Wikipedia behauptet das Gegenteil)
\end{itemize}
\begin{itemize}
  \renewcommand{\labelitemi}{+}%
  \item Lautsprecher fast beliebig positionierbar
\end{itemize}

\subsubsection*{Wellenfeldsynthese}
\begin{itemize}
  \item Ziel: korrektes Klangfeld im gesamten Raum, damit überall ideale Hörposition
  \item ``Array'' von vielen direkt nebeneinander angeordneten Lautsprechern, die durch phasenkorrekte Ansteuerung eine gemeinsame Wellenfront erzeugen (Huygensches Prinzip)
\end{itemize}
\begin{itemize}
  \renewcommand{\labelitemi}{\(-\)}%
  \item nichts für's Heimkino: Aufwändige Installation und viele Lautsprecher
  \item der Hörraum darf den Klang nicht zusätzlich beeinflussen (unmöglich, aber näherungsweise durch kleine, möglichst absorptive Räume)
  \item üblicherweise Positionierung nur in horizontaler Ebene durch horizontales umlaufendes Lautsprecherband
  \item ist das Lautsprecherband unterbrochen, entstehen dort Störungen
\end{itemize}
\begin{itemize}
  \renewcommand{\labelitemi}{+}%
  \item ``kompletteste'' Erzeugung eines vollständigen Klangfeldes
  \item überall ideale Hörposition
\end{itemize}

\subsubsection*{Psychoakustik}
``externes Ohr'' meint Einfluss von Körper und Ohrmuschel auf den Klang.
\begin{itemize}
  \item Ausnutzen der physiologischen Eigenschaften der 3D-Wahrnehmung des Ohrs
  \item Aufnahme z.B.\ mit Kunstkopf
  \item künstliche Erzeugung des Gesamtsignals durch Messen der Charakteristik des Kunstkopfes und verrechnen der virtuellen Klangquellen mit den gemessenen Daten (z.B. HRTF)
\end{itemize}
\begin{description}
  \item[Real:] Sound -- reales externes Ohr -- Hörkanal
  \item[Binaural:] Sound -- virtuelles externes Ohr -- Hörkanal (z.B. per Kopfhörer!)
  \item[Transaural:] Sound -- virtuelles externes Ohr -- reales externes Ohr -- Hörkanal (z.B. Stereo-Lautsprecher)
    \begin{itemize}
      \item Problem des Übersprechens: Linkes Ohr hört auch Signal für rechtes Ohr und umgekehrt
      \item Idee: ``Gegenschall'' zum rechten Signal auf linkes Signal legen (verzögert)
      \item neues Problem: für linkes Ohr bestimmter Gegenschall komm auch beim rechten Ohr an...
    \end{itemize}
\end{description}



\section{Haptik}


\subsection[Wahrnehmung]{Haptische Wahrnehmung}
\begin{itemize}
  \item Exterozeption und Propriozeption
  \item Anwendung bspw.\ für Simulation oder Fernausführung manueller Tätigkeiten
\end{itemize}
--- Aufzählung von Referenzwerten bzgl.\ der Wahrnehmung wird hier ignoriert ---\\\\
Nachzubildende Ereignisse:
\begin{itemize}
  \item Kontakt
  \item Ergreifen
  \item Abtasten
\end{itemize}
Dazu müssen folgende Parameter simuliert werden:
\begin{description}
  \item[Viskosität] eines Mediums
  \item[Oberflächenreibung] als Haft- und Gleitreibung
  \item[Oberflächenstärke] als Widerstandskraft eine Oberfläche gegen Eindringen eines Objekts
\end{description}


\subsection[Simulation]{Haptik-Simulation}

\subsubsection*{Grundlagen}

\begin{itemize}
  \item ``Berechnung und Darstellung von virtuellen mechanischen Kräften auf einem haptischen Device''
  \item Anforderung: 1kHz Framerate
\end{itemize}

Ansätze:
\begin{description}
  \item[Penalty-Methode:] Rückwirkende Kraft hängt nur von Eindringtiefe in Objekt ab\\
    Extrem grobe Simulation von Oberflächen, Interaktor wird einfach aus Objekt herausgedrängt
  \item[Dynamische Methode:] Kraft abhängig von Eindringhistorie
    \begin{description}
      \item[Impedance-Ansatz:] Meistverwendeter Ansatz
        \begin{enumerate}
          \item haptisches Gerät liefert \emph{Positionsinformation}
          \item Simulation berechnet Rückwirkungskräfte
        \end{enumerate}
      \item[Admittance-Ansatz:]~
        \begin{enumerate}
          \item haptisches Gerät liefert \emph{Kräfte des Benutzers}
          \item Simulation berechnet neue Position des Devices
        \end{enumerate}
    \end{description}
\end{description}
Haptik-Loop:
\begin{enumerate}
  \item erfrage Position, Geschwindigkeit und/oder Kräfte
  \item Vergleich mit Objekten der Szene
  \item Berechnung der rückwirkenden Kraft
  \item Benutzerinter- und Reaktion
\end{enumerate}
Feedback-Möglichkeiten:
\begin{itemize}
  \item Widerstand bzw. Gegenkräfte
  \item elektrische Stimulation der Haut
  \item Ultraschall oder rotierende Scheiben (Oberflächenbeschaffenheit)
  \item Vibrationen
    \begin{description}
      \item[Zweck:] Fühlen von Oberflächenstrukturen
      \item[Problem:] Frequenzen oberhalb von 15 kHz benötigt
    \end{description}
\end{itemize}

\subsubsection*{Phantom Devices}
\begin{itemize}
  \item Hardware für Kraft-Feedback
  \item Nutzer bekommt stiftförmigen Griff in die Hand
  \item Simulation von Kräften in allen 6 Freiheitsgraden
\end{itemize}

\subsubsection*{Penalty Based Rendering}
\begin{itemize}
  \item Kraft proportional zur Eindringtiefe
  \item wirkt in Richtung des nächsten Oberflächenpunktes
\end{itemize}
\begin{itemize}
  \renewcommand{\labelitemi}{\(-\)}%
  \item Eindringtiefe nicht eindeutig (welche Richtung?)
  \item bei dünnen Objekten wird der HIP (Haptics Interface Point) eventuell ab bestimmter Eindringtiefe durch das Objekt hindurch gedrängt (wenn näher an gegenüberliegender Oberfläche...)
\end{itemize}

\subsubsection*{Proxy Rendering (dynamischer Ansatz)}
\begin{itemize}
  \item Idee: Richtung der Eindringung soll bekannt sein
  \item Proxy-Objekt: Massenlose Kugel
  \item Proxy dringt (vorerst) nicht in Oberfläche ein, versucht aber, dem HIP zu folgen
  \item Distanz zwischen HIP und Proxy bestimmt Kraft (``virtuelle Feder'')
\end{itemize}
Für \emph{Volumen-Rendering} braucht man ein 3D-Array mit Dichte, Materialklassifikation, Steifheit (Federkonstante!) und Viskosität. Oberflächen werden hier durch hohe Dichteunterschiede dargestellt (großer Gradient) (?)
\begin{description}
  \item[Oberflächendurchdringung]~
    \begin{itemize}
      \item Schwellwert für Kraft in Richtung der Oberflächennormalen, ab der der Proxy eindringt
      \item Eindringung findet nur in Richtung der Oberflächennormalen statt, der tangentiale Anteil wird über die Reibungskomponente berechnet
      \item Positionsänderung wird um Durchdringungsstärke vermindert
    \end{itemize}
  \item[Haftreibung]~
    \begin{itemize}
      \item Schwellwert für Kraft in zur Oberfläche tangentialer Richtung, ab der Proxy sich entlang der Oberfläche bewegt
      \item Abhängig von Kraft in Richtung der Normalen (Anpressdruck) und Reibungskonstante
      \item Positionsänderung wird um Haftungsstärke vermindert
    \end{itemize}
  \item[Viskosität]~
    \begin{itemize}
      \item Schwellwert?
      \item Viskosität erforderlich für Widerstand in Bereichen ohne Gradienten (Oberflächen)
      \item wirkt stets der vorher berechneten Bewegungsrichtung entgegen (verlangsamt)
    \end{itemize}
\end{description}
\begin{itemize}
  \renewcommand{\labelitemi}{\Large{!}}%
  \item Berechnung der Gradienten sollte als Vorverarbeitungsschritt durchgeführt werden
  \item Reibungskomponente existiert nur, wenn Proxy auf Oberfläche
\end{itemize}



\section{Echtzeitschatten}


\subsection{Grundlagen}
\begin{itemize}
  \item Schatten wichtig für Raumeindruck, Relativpositionen der Objekte
  \item ``echte'' Schattierung nicht in Rendering-Pipeline lösbar, da auf Objekte der Szene zugegriffen werden muss
  \item Echtzeitschatten deshalb ``getrickst'', ``Fakes'' statt globalem Beleuchtungsmodell, das in der RP nicht umsetzbar ist
\end{itemize}
Algorithmenklassen für Schatten:
\begin{description}
  \item[Scanline/Raytracing]~
    \begin{itemize}
      \item Strahlverfolgung vom Augpunkt in die Szene bis zur ersten (teil-)diffusen Fläche
      \item Prüfe Verdeckung des Punktes gegenüber der Lichtquelle mittels ``Shadow Ray''\\\(\Rightarrow\) Schatten
    \end{itemize}
    \begin{itemize}
      \renewcommand{\labelitemi}{\(-\)}%
      \item langsam
    \end{itemize}
  \item[Global Illumination]~
    \begin{itemize}
      \renewcommand{\labelitemi}{\(-\)}%
      \item sehr aufwändig, nicht echtzeitfähig
    \end{itemize}
    \begin{itemize}
      \renewcommand{\labelitemi}{+}%
      \item beste visuelle Qualität
    \end{itemize}
  \item[Lokale Beleuchtung, Rendering Pipeline]~
    \begin{itemize}
      \item Verdeckungen zur Lichtquelle vorberechnen und mit dem lokalen Beleuchtungsmodell kombinieren
    \end{itemize}
    \begin{itemize}
      \renewcommand{\labelitemi}{\(-\)}%
      \item schlechte Qualität
      \item Berechnung der Verdeckung aufwändig, langsam
      \item ? Meshing
    \end{itemize}
  \item[Schatten als geometrisches Objekt]~
    \begin{itemize}
      \item meistverwendete Algorithmen für Echtzeitschatten
    \end{itemize}
\end{description}


\subsection{Projective Shadows}
Performant bei vielen schattenwerfenden und wenig schattierten Objekten\\
Vorgehensweise:
\begin{enumerate}
  \item projiziere schattenwerfende Objekte aus Sicht der Lichtquelle auf schattierte Objekte
  \item rendere in Texturen (eine pro schattiertem Objekt, auch bei mehreren schattenwerfenden!)
  \item rendere Szene normal, mappe Schattentexturen auf schattierte Objekte
\end{enumerate}
\begin{itemize}
  \renewcommand{\labelitemi}{\(-\)}%
  \item Texturauflösung begrenzt Schattenauflösung
\end{itemize}
\begin{itemize}
  \renewcommand{\labelitemi}{+}%
  \item Punktlichtquellen und paralleles Licht möglich durch verschiedene Projektionen (orthogonal vs.\ perspektivisch)
\end{itemize}
Alternative Methode: Geometrie des schattenwerfenden Objekts wird direkt auf das schattierte projiziert (ohne Umweg über Textur)
\begin{enumerate}
  \item Zeichne Objekt, auf das der Schatten gerendert werden soll
  \item projiziere schattenwerfendes Objekt durch \emph{shadow projection matrix} auf das zu schattierende Objekt. Die Matrix ist vom schattierten Objekt und der Lichtquellenposition abhängig
\end{enumerate}
\begin{itemize}
  \item Problem: z-Fighting zwischen schattiertem Objekt und Schatten
  \item Lösungsidee 1: Zeichnen der Schatten ``etwas näher'' Richtung Augpunkt
  \item Lösungsidee 2:
    \begin{enumerate}
      \item Markieren der Schattenpixel im Stencil Buffer
      \item Schatten ohne Depth-Test an im Stencil Buffer markierte Stellen rendern
    \end{enumerate}
  \item Problem: Schatten sind absolut schwarz bzw.\ einfarbig
  \item Lösung: Verknüpfen von im Stencil Buffer markierten Schattenpixeln mit Hintergrund durch Blending
  \item Komplexität alternative Methode: Projektionsmatrizen \(\times\) Lichtquellen
\end{itemize}

\subsection{Shadow Volumes}
Idee:
\begin{itemize}
  \item geometrische Definition des Raumes, der im Schatten liegt durch Polygonliste
  \item liegt ein Oberflächenpixel in diesem Raum, wird es abgedunkelt (schattiert) dargestellt
\end{itemize}
Vorgehensweise:
\begin{enumerate}
  \item berechne über Flächennormalen die Silhouette des schattenwerfenden Objekts (Skalarprodukt)
  \item erzeuge unendlich lange Strahlen von der Lichtquelle durch alle Vertices der Silhouette\\
    Diese definieren die Kanten des Schattenvolumens
  \item berechne Anzahl der Polygone des Schattenvolumens zwischen Augpunkt und Oberflächenpixel
  \item ungerade = schattiert, gerade = nicht schattiert\\
    umgekehrt wenn Augpunkt innerhalb des Schattenvolumens
\end{enumerate}
Hardware-basierte Auswertung:
\begin{enumerate}
  \item Rendere Szene (von hinten nach vorne wegen z-Buffer)
  \item addiere Polygone des Shadow Volume auf Stencil Buffer
  \item Wenn ``schattenempfangende'' Objekte gerendert werden, schattiere entsprechende Pixel, wenn deren Stencil-Buffer-Eintrag ungerade ist
\end{enumerate}
\begin{itemize}
  \renewcommand{\labelitemi}{\(-\)}%
  \item schwierig für nichtpolygonale Objekte
  \item Silhouetten finden allgemein nicht einfach, fehlerbehaftet
  \item Neuberechnung der Volumes notwendig, wenn Lichtquellen oder schattenwerfende Objekte sich bewegen
  \item Speicheraufwand für Shadow-Geometrien
\end{itemize}
\begin{itemize}
  \renewcommand{\labelitemi}{+}%
  \item ?
\end{itemize}


\subsection{Shadow Maps}
Vorgehensweise:
\begin{enumerate}
  \item rendere Szene von der Lichtquelle aus auf nicht sichtbaren Framebuffer\\
    (der Tiefenpuffer ist die Shadow Map)
  \item Bestimmung der Schattiertheit eines Pixels beim Rendern der Szene:
    \begin{enumerate}
      \item Projiziere 3D-Pixelkoordinaten in Shadow-Map-Projektion
      \item wenn der z-Wert des projizierten Pixels größer ist als der des entsprechenden Pixels auf der Shadow Map, ist das Pixel von der Lichtquelle aus verdeckt --- also schattiert
    \end{enumerate}
\end{enumerate}
\begin{itemize}
  \renewcommand{\labelitemi}{\(-\)}%
  \item Shadow Maps sind diskretisiert (Auflösungsartefakte) und quantisiert (Genauigkeit des Tiefenpuffers)
\end{itemize}
\begin{itemize}
  \renewcommand{\labelitemi}{+}%
  \item allgemeine Methode für alle Schatten, auch \emph{self shadowing}
  \item Schatten-Repräsentation unabhängig von Komplexität der Szene (außer Anzahl Lichtquellen)
\end{itemize}


\subsection{Soft Shadows}
\begin{itemize}
  \item Ziel: Emulation flächiger Lichtquellen für weiche Schattengrenzen
  \item Algorithmus:\\
    Rendere Schatten mehrmals mit leicht verschobener Lichtquelle und addiere Schatten (Blending)
  \item Gleichmäßiges Sampling der leuchtenden Fläche
\end{itemize}


\subsection{Ambient Occlusion}
\begin{description}
  \item[Idee:] Schattierung eines Pixels proportional zum Grad der Verdeckung gegenüber einer theoretischen gleichmäßigen Umgebungsbeleuchtung
  \item[Algorithmus:]~
    \begin{enumerate}
      \item Schiesse Strahlen vom Objektpunkt in ``die Umgebung''
      \item Zählen ``verdeckter'' Strahlen
    \end{enumerate}
\end{description}
Implementierung in Hardware:
\begin{description}
  \item[Idee:] Nur Objekte in nächster Umgebung signifikant
  \item[Verfahren:]~
    \begin{itemize}
      \item teste z-Werte von Pixeln in direkter Umgebung des betrachteten Pixels im Framebuffer
      \item Wähle Pixel entsprechend einer Maske aus, die ein gleichmäßiges Sampling sicherstellt
    \end{itemize}
\end{description}


\end{document}
